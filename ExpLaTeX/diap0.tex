\begin{frame}
    \frametitle{Compilando \LaTeX{}}

    \begin{columns}[onlytextwidth]
        \begin{column}{0.48\textwidth}
            \centering
            \begin{exampleblock}{Editable} % MARCO width=8cm, height=3cm
                \begin{center}
                    \begin{enumerate}                        
                        \item { \ttfamily \$ latex informe.tex}
                        \item { \ttfamily \$ pdflatex informe.tex}
                    \end{enumerate}                                     
                \end{center}
            \end{exampleblock}
            \begin{enumerate}
                \item { \ttfamily informe.aux  informe.dvi  informe.log}
                \item { \ttfamily  informe.aux  informe.log  informe.pdf}
            \end{enumerate}
            
            %\rule{100pt}{150pt}% Place your graphic here
        \end{column}
        
        \begin{column}{0.46\textwidth}
            \begin{exampleblock}{} %marco tipo block
                \begin{itemize}
\item  Relación de uso más indicada cuando es necesaria una asociación entre clases pero el \textit{principio de Liskov} no se cumple
\item  Herencia múltiple manual (composición de objetos). 
%\includegraphics[width=\textwidth]{Caja.png}
\item  Usado en Patrones de diseño como \textit{Strate, Strategy, Visitor}

\item %Caracterizado por la reutilización selectiva
Utilizando interface, la delegación puede hacerse de manera más flexible y fuertemente tipada
\end{itemize}

\end{exampleblock}
\end{column}
\end{columns} 
\end{frame}

