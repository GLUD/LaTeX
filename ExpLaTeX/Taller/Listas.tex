\begin{frame}
    \frametitle{Listas en \LaTeX{}}

    \begin{columns}[onlytextwidth]
        \begin{column}{0.48\textwidth}
            \centering
            \begin{exampleblock}{Con un guión definido} % MARCO width=8cm, height=3cm
                \begin{itemize}
                    \item[$*$] Madrid.
                    \item Castilla la Mancha.
                    \item Castilla y León.
                    \begin{itemize}
                        \item Segovia.
                        \item Ávila.
                    \end{itemize}
                \end{itemize}
            \end{exampleblock}
        \begin{exampleblock}{Con un guión definido} % MARCO width=8cm, height=3cm    
            \begin{enumerate}[1.]
                \item Manzanas.
                \item Plátanos.
                \item Pescado fresco.
                \begin{enumerate}[a)]
                    \item Emperador.
                    \item Gallo.
                \end{enumerate}
            \end{enumerate}
        \end{exampleblock}      
            %\rule{100pt}{150pt}% Place your graphic here
        \end{column}
        
        \begin{column}{0.46\textwidth}
            \begin{exampleblock}{} %marco tipo block
                \begin{description}
                    \item[Australia:] Canguro.
                    \item[EEUU:] Águila calva.
                    \item[España:] Toro.
                    \item[México:] Águila real.
                \end{description}
            \end{exampleblock}
        \end{column}
    \end{columns} 
\end{frame}


% \usepackage{hyperref}
%   para colocar hipervinculos
%\href{url}{http://www.texample.net/tikz/resources/}

%   Above = antedicho, encima, arriba, encima de


\begin{comment}
%-------------- ENUMERACION

\begin{itemize}
    \item[$*$] Madrid.
    \item Castilla la Mancha.
    \item Castilla y León.
    \begin{itemize}
        \item Segovia.
        \item Ávila.
    \end{itemize}
\end{itemize}
___________________________________

\begin{enumerate}[1.]
    \item Manzanas.
    \item Plátanos.
    \item Pescado fresco.
    \begin{enumerate}[a)]
        \item Emperador.
        \item Gallo.
    \end{enumerate}
\end{enumerate}

__________________________________-

\begin{description}
    \item[Australia:] Canguro.
    \item[EEUU:] Águila calva.
    \item[España:] Toro.
    \item[México:] Águila real.
\end{description}

\end{comment}




%  \includegraphics[width=\textwidth]{ //NOMBRE DE LA IMAGEN }   == linea para añadir imagenes gracias a la librería.

% "  " \par aplica Sangría

%  \textcolor{red}{Hola} == coloca el texto en color.


% "En el mundo del código abierto, la vergüenza es uno de los principales factores de motivación hacia el aumento de la calidad de su código!)" 
% página 34, Software libre para una sociedad libre
